\documentclass[12pt,a4paper]{article}
%\usepackage{xcolor} \pagecolor[rgb]{0.5,0.5,0.5} \color[rgb]{1,1,1}
%\usepackage[utf8]{inputenc}
\usepackage[shortlabels]{enumitem}
%\usepackage{bm}
\usepackage{mathrsfs}
\usepackage{amsmath}
\usepackage{amssymb}
\usepackage{hyperref}
\begin{document}
\title{Solutions to BDA Assignment 2,
 2020/2021 Semester 2}
\author{ YOUR NAME, STUDENT ID}
\date{\today}
\maketitle

IMPORTANT: EVERY ANSWER NEEDS TO HAVE AT LEAST ONE SENTENCE OF EXPLANATION CLEARLY DEMONSTRATING THAT YOU UNDERSTAND WHAT YOU ARE DOING.
SOLUTIONS WITHOUT EXPLANATION WILL BE MARKED AS 0, IRRESPECTIVELY WHETHER THE COMPUTATIONS ARE CORRECT OR NOT.

FOR THE RESULTS, YOU CAN COPY PASTE THE PLOTS AND NUMBERS FROM KAGGLE. THE SOLUTION HAS TO BE CLEARLY EXPLAINED AND READABLE WITHOUT LOOKING AT YOUR CODE.

BESIDES SUBMITTING THIS REPORT, YOU ARE ALSO REQUIRED TO SUBMIT YOUR IPYNB NOTEBOOK FROM KAGGLE. YOUR CODE HAS TO BE ABLE TO RUN THROUGH ALL QUESTIONS AND REPRODUCE EVERY RESULT IN THIS REPORT.

\vspace{1cm}

\noindent\textbf{1)}
\begin{enumerate}[(a)]
\item
Explanation:

Results:

\item
Explanation:

Results:

\item
Explanation:

Results:

\item
Explanation:

Results:

\item
Explanation:

Results:

\end{enumerate}

\noindent\textbf{2)}
\begin{enumerate}[(a)]
\item
Explanation:

Results:

\item
Explanation:

Results:

\item
Explanation:

Results:

\item
Explanation:

Results:

\item
Explanation:

Results:


\end{enumerate}
\end{document}